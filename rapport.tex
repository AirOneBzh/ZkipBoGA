\documentclass[10pt,a4paper]{article}


\usepackage[utf8]{inputenc}
\usepackage[T1]{fontenc}
\usepackage[french]{babel}
\usepackage{geometry}
\geometry{hmargin=2.5cm,vmargin=1.5cm}

\title{\textbf{ZkipBoGA} Le jeu de SkipBo en C}
\author{Erwan LE CORNEC, Camil Brahmi}
\date{}

\begin{document}
\maketitle
\newpage
\paragraph{Règles}

Vous avez droit à 5 cinq cartes en main piochées (automatiquement) au début du tour ainsi qu'à 5 piles de défausse personnelle pour monter une pile en valeur et vous approcher de celle de votre tas.
Pour monter la valeur d'une pile vous devez poser une carte immédiatement successive la pile commence  à 1 et se termine à 12.

Si vous n'avez pas la carte de valeur nécessaire il vous est possible d'utiliser la carte joker qui remplace toutes les autres.

Si vous n'avez plus de cartes en main piochez en cinq (automatiquement).
Votre tour ne s'arrête alors que lorsque vous ne pouvez plus ou ne voulez plus poser de cartes.

Vous terminez votre tour en posant une carte dans une de vos piles de défausse.

\end{document}
